\chapter{Introduction}

Android operating system is an open source operating system currently developed by Google. The source code is released by Google under open source licenses. This open architecture made android popular and applications are developed by the huge community of developers. These applications help to enlarge the capabilities and functionalities of the Android operating system.  Apart from making telephone calls and sending SMS, mobile devices now help to keep in touch with our social networking, messaging, e-commerce,  video calling, check emails, collaborate and share ideas with other peoples etc. Most of the android users are not familiar with personal computer or programming languages and they don't know what is inside the OS. Android is running on top of Linux Kernel. A large portion of it is written in C and C++. The application framework is written in Java.  Android is a full-fledged networking device. A clever attacker most of the time targets the coding vulnerabilities. So it should satisfy all the secure coding guidelines 
%Self reading over. There is no critical bug
%PG- 0%
%Author @rahul
\section{Secure Software}
The primary goal of software
security is to maintain the confidentiality, integrity, and availability. This aim is fulfilled via the implementation of security guidelines. Developing a secure software requires a good understanding of security guidelines.
A fundamental component of secure software development is well-documented coding standards that support programmers to follow a uniform set of rules and
recommendations. It will tell what he can do and what he can't. Once established, these rules and
recommendations can be used as a metric
to assess source code\cite{nist}
%Self reading over. There is no critical bug
%PG- 20%
%Author @rahul
\newpage
\section{Problem Statement}

Conformance to secure coding standard requires that the code not contains any violations of the rules specified in a particular coding standards. If an exceptional condition is claimed, the exception must correspond to a predefined exceptional condition, and the application of this exception must be documented in the source code\cite{cert-c}. Coding standards are not common before last five years so that developers are not that much bother about secure coding guidelines. Most of the existing projects are vulnerable because of this bad coding practice.This thesis will address the bad coding practice in the Android Open Source Project.
%Self reading over. There is no critical bug
%PG- 20%but properly cited
%Author @rahul
\section{Motivation}
Android is the most widely used platform for smart phones. It gives support for a variety
of file systems, process management, networking, virtual memory, root access(if rooted) and so on. In short, Android has the capability of a full operating system.  This means the attack surface is wide. Moreover, Android users are likely to be computer illiterate. So it should be free of coding vulnerabilities.
%Self reading over. There is no critical bug
%PG- 0
%Author @rahul

\section{Aim}
A proactive mobile operating system that satisfies a maximum number of security standards. Normally an end user won't understand what standard is there and what security it is providing but
this will close the back door for the attacker that who uses the coding vulnerabilities for penetration   
%Self reading over. There is no critical bug
%PG- 0
%Author @rahul
\section{Organization}
The rest of the thesis is organized as follows: Chapter 2 provides background work undergone for the
thesis. Chapter 3 contains the Proposed system. Chapter 4 contains the Related works did for the thesis.
Chapter 5 Explains a Details of a solution. Chapter 6 contains the Implementation and result. Chapter 7 contains Conclusion and Future work of the thesis.
% -eof-
%Self reading over. There is no critical bug
%PG- 0
%Author @rahul