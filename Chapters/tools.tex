\chapter{Recommended Software Tools}

Only open-source tools are mentioned in this section.  Only a few
links are given below because they have been changing too rapidly.
Search the web.

\begin{description}
\item[tetex]There are multiple distributions of TeX and LaTeX for
  Linux.  The distribution known as the tetex is recommended.
\item[MiKTeX] is a complete \TeX{} distribution for Windows.
\item[lyx] and {\tt kile} are frontends to \LaTeX{} giving a WYSIWIG
  view.  You might like them.

\item[{\tt emacs}] is excellent for editing TeX files.  There are
  style files for TeX.  Make sure they are loading when you begin
  editing a file with the extension .tex.  Visit
  \url{http://www.emacswiki.org/}.

\item[{\tt pdflatex}]
directly produces .pdf files.  However, often it is more
convenient to produce .dvi, then .ps, and finaly .pdf.  DVI-to-PS is
by dvips, PS to PDF is by ps2pdf.

\item[xfig] is a GUI tool for vector-based drawing.  It can export
  into \LaTeX{}, eps and other formats.

\item [OpenOffice] and {\tt Inkscape} have vector drawing components
  that are quite good.  They can export into eps or pdf.

\item [{\tt metapost},] {\tt asymptote} are programming languages for
meticulous drawing.

\item[{\tt graphviz}] is a layout tool for graph theoretic graphs.

\item[{\tt tth}] generates {\sc html} files from your LaTeX files.
  The {\tt tex2page} is better but requires the programming language
  Scheme.

\item[kdvi]  xdvi, xpdf and okular are for previewing.

\end{description}

% -eof-
