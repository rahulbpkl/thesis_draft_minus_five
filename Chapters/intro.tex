\chapter{Introduction}

Android operating system is an open source operating system currently developed by Google. The source code is released by Google under open source licenses. This open architecture have nurtured the growth of applications developed by the huge community of developers. These applications helps to extend the capabilities and functionalities of the Android operating system. From a device used only for telephony and messaging, Internet connectivity helped Mobiles devices to evolve and become a crucial part of our daily lives. Apart from making telephone calls and sending SMS mobile devices now helps to keep in touch with our social networking services, read news, check emails, do e-commerce, collaborate and share ideas with other peoples etc. Android is built on top of Linux Kernel, and a large portion of it is written in C and a good number in C++ also. Most of the application framework is written in Java.  Android is a full-fledged networking device. A cleaver attacker most of the time targets the coding vulnerabilities. So it should satisfy all the secure coding guidelines. 
\section{Secure Software}
Building secure software requires a basic understanding of security guidelines.The goal of software security is to maintain the confidentiality, integrity, and availability of information resources in order to enable successful operations. This goal is accomplished through the implementation of security controls.  An essential element of secure software development is well-documented and enforceable coding standards. Coding standards encourage programmers to follow a uniform set of rules and guidelines determined by the requirements of the project and organization, rather than by the programmer’s familiarity or preference. Once established, these standards can be used as a metric to evaluate source code.\cite{nist}
\newpage
\section{Problem Statement}

Conformance to secure coding standard requires that the code not contains any violations of the rules specified in a particular coding standards. If an exceptional condition is claimed, the exception must correspond to a predefined exceptional condition, and the application of this exception must be documented in the source code\cite{cert-c}. Coding standards are not common before last five years so that developers are not that much bother about secure coding guidelines. Most of the existing projects are vulnerable because of this bad coding practice.This thesis will address the bad coding practice in the Android Open Source Project.
\section{Motivation}
Android is the most widely used platform for smart phones. Android
provides support for a variety of file systems, has a full network layer,
has a full process management and virtual memory, root access(if rooted) and so on. In short, Android has the capability of a full operating system.
But, this also means that the attack surface is wide. Also, Android user is likely to be computer illiterate, may not have used a
computer system ever.
\section{Aim}
A proactive mobile operating system that satisfies a maximum number of security standards. Normally an end user won't understand what standard is there and what security it is providing but this will close the back door for the attacker that who uses the coding vulnerabilities for penetration.   
\section{Organization}
The remainder of the thesis is organized as follows: Chapter 2 provides background work undergone for the
thesis. Chapter 3 Proposed system. Chapter 4 Related works.
Chapter 5 Details of a solution. Chapter 6 Implementation and result. Chapter 7 Conclusion and Future work.
% -eof-
