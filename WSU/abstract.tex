\newpage
\thispagestyle{plain}

{\centering\bf ABSTRACT\\}\par\vskip 2cm

\singleSpacing
\noindent
Subramanian, Sripriya.          %% last, first name, upper-lower case mix
M.S.  Department of Computer Science and Engineering,
Wright State University,
2003.                           %% this year
Sniffing the Ethernet with High Quality Tools. %% title

\par\vskip 2cm

\doubleSpacing

[This document is an example collection of files intended to help my
students in using LaTeX as they prepare their theses.  An abstract is
typically about one page.  The following is an example of an abstract
of a thesis.]

This thesis is a study of software quality in the limited context of a
class of network software tools called {\em sniffers}. Sniffers are
network monitoring tools used in the administration of network
security. We analyzed five of the hundreds of the existing sniffers to
determine the causes of poor quality and the methods to eliminate the
problems. We subjected the five selected sniffers to both manual
analysis and analysis by software quality assessment tools. We
classify the 1000+ errors so discovered in these sniffers. Based on
these results, we designed and implemented a new sniffer that is
intended as a model of high quality sniffer. The methods applied to
analyze and enhance the quality of software are studied.

Quality assessment software tools fall into two categories: 1) Static
checkers and 2) Dynamic checkers. Static checkers analyze the source
code. Dynamic checkers stand as guards during run-time. We have
collected and used exhaustively the checker tools that are in the open
source archives. In the course of our use, the quality of the software
quality tools is itself analyzed.

This thesis contributes several case studies of open source software
projects. It also contributes a new distributed sniffer to the open
source. The distributed sniffer can monitor multiple networks and
output desired packet details. We designed our sniffer to include a
MySQL based collector program, packet capture programs and viewer
programs. Our sniffer supports multiple capture and viewer
programs. We applied software engineering methods to eliminate the
quality issues associated with software. We use the results of our
analysis of existing sniffers in avoiding poor quality in our
sniffer. We prepared our sniffer for manual audit using documentation
tools. This new sniffer is an example of {\it literate programming}
that is worthy of study by software engineering students.

\newpage




